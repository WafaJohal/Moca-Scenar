\usepackage[a4paper,inner=3.8cm,outer=2.6cm,bottom=3.7cm,top=3.7cm]{geometry} %
\DeclareFixedFont{\RM}{T1}{ptm}{b}{n}{2cm}

 %%%%%%%%%%%%%%%%%%%%%%%%%WAFA COMMAND%%%%%%%%%%%%%%%%%%%%%%%%%%ù
% abbreviation list 
\usepackage[]{nomencl}
\makenomenclature
 
\makeatletter
\def\cleardoublepage{\clearpage\if@twoside \ifodd\c@page\else
 \hbox{}
 \thispagestyle{empty}
 \newpage
 \if@twocolumn\hbox{}\newpage\fi\fi\fi}
\makeatother
%%%%%%% Pagestyle stuff %%%%%%%%%%%%%%%%%
 \usepackage{fancyhdr}                 %%
   \pagestyle{fancy}                   %%
   \fancyhf{} %delete the current section for header and footer
    \columnsep 3em                     %%
%    \renewcommand{\headrulewidth}{0.2pt}
 %   \renewcommand{\footrulewidth}{0.2pt}
    
  %  \renewcommand{\headrule}{{\color{blue}\hrule width\headwidth
   %     height\headrulewidth \vskip-\headrulewidth}}

    \renewcommand{\chaptermark}[1]{\markboth{#1}{}}
    \renewcommand{\sectionmark}[1]{\markright{#1}{}}

    \fancyhead[LE,RO]{\thepage}
    \fancyhead[LO]{\thesection \hspace{2pt} \rightmark}         %%    
    \fancyhead[RE]{\textsc{\leftmark}}
    

    %\fancyfoot[C]{Wafa \textsc{Benkaouar}}
    %\fancyfoot[LE,RO]{\thepage}


%%%%%%% End Pagestyle stuff %%%%%%%%%%%%%

% \makechapterstyle{BlueBox}{%
% \renewcommand{\chapnamefont}{\large\scshape}
% \renewcommand{\chapnumfont}{\Huge\bfseries}
% \renewcommand{\chaptitlefont}{\raggedright\Huge\bfseries}
% \setlength{\beforechapskip}{20pt}
% \setlength{\midchapskip}{26pt}
% \setlength{\afterchapskip}{40pt}
% \renewcommand{\printchaptername}{}
% \renewcommand{\chapternamenum}{}
% \renewcommand{\printchapternum}{%
% \sbox{\ChpNumBox}{%
% 46
% \BuildChpNum{\chapnamefont\@chapapp}%
% {\chapnumfont\thechapter}}}
% \renewcommand{\printchapternonum}{%
% \sbox{\ChpNumBox}{%
% \BuildChpNum{\chapnamefont\vphantom{\@chapapp}}%
% {\chapnumfont\hphantom{\thechapter}}}}
% \renewcommand{\afterchapternum}{}
% \renewcommand{\printchaptertitle}[1]{%
% \usebox{\ChpNumBox}\hfill
% \parbox[t]{\hsize-\wd\ChpNumBox-1em}{%
% \vspace{\midchapskip}%
% \thickhrulefill\par
% \chaptitlefont ##1\par}}%
% }
% \chapterstyle{BlueBox}

%%%% START CUSTOMIZING CHAPTER STYLES %%%%%

% Changement de la syntaxe de la commande \chapter
\usepackage{helvet}
\usepackage{psboxit,pstcol,graphicx}
\usepackage{pslatex}

\makeatletter
\makeatletter
\def\@makechapterhead#1{%
  \parindent \z@  \reset@font
% \vskip 50pt
  %\hrulefill
 % \vskip 20pt
  \hbox to \hsize{%
    \rlap{\raisebox{-0.1em}{\raisebox{ \depth}{}}}%
          %\includegraphics[width=10em]{images/logo3.png}}}}%
    \rlap{\raisebox{0.1em}{\hbox to 10em{\hss \hspace{0.5em}
    \reset@font\fontsize{6em}{6em}\selectfont  \thechapter
\hss}}}%
    \hspace{10em}%
    \vbox{%
      \advance\hsize by -10em
\raggedright
      \reset@font\bfseries\Huge
      #1
      \par
      }%
    }%
%   \vskip 20pt
%   \hrulefill
   \vskip 50pt
  }
\def\@makeschapterhead#1{%
  \parindent \z@ \raggedright \reset@font
% \vskip 10pt
%   \hrulefill
%   \vskip 20pt
  \hbox to \hsize{%
    \rlap{\raisebox{-2.5em}{\raisebox{\depth}{}}}%
         %\includegraphics[width=10em]{images/logo3.png}}}}%
    \hspace{10em}%
    \vbox{%
      \advance\hsize by -10em
\raggedright
      \reset@font\bfseries\Huge
      #1
      \par
      }%
    }%
%   \vskip 20pt
%   \hrulefill
   \vskip 50pt
}


% \makeatletter
% \def\thickhrulefill{\leavevmode \leaders \hrule height 1ex \hfill \kern \z@}
% \def\@makechapterhead#1{%
%   \vspace*{10\p@}%
%   {\parindent \z@ \centering \reset@font
%         \thickhrulefill
%         \par\nobreak
%         \scshape \@chapapp{} \strut\thechapter
%         \par\nobreak
%         \interlinepenalty\@M
%         \hrule
%         \vspace*{10\p@}%
%         {\Huge \bfseries #1}\par\nobreak
%         \thickhrulefill
%         \vspace*{10\p@}%
%     \vskip 100\p@
%   }}
% \def\@makeschapterhead#1{%
%   \vspace*{10\p@}%
%   {\parindent \z@ \centering \reset@font
%         \thickhrulefill
%         \par\nobreak
%         {\Huge \bfseries \strut #1}\par\nobreak
%         \interlinepenalty\@M
%         \hrule
%         \vspace*{10\p@}%
%     \vskip 100\p@
%   }}
% \makechapterstyle{section}{
% \renewcommand{\printchaptername}{}
% \renewcommand{\chapternamenum}{}
% \renewcommand{\chapnumfont}{\normalfont\Huge\bfseries}
% \renewcommand{\printchapternum}{\chapnumfont \thechapter\space}
% \renewcommand{\afterchapternum}{}
% }

%%%%%%%%%%%%%% STOP CUSTOMIZING CHAPTER STYLES %%%%%

%% Section Style


\usepackage{lipsum}
\usepackage{textpos}
\usepackage{graphics}       % for jpeg images
\usepackage{algorithmic}    % Algorithms
%\usepackage[T1]{fontenc}    % For some european characters
 \usepackage{etex}          % For some reason, pdflatex breaks if I don't include the etex package
 \usepackage{amsmath}       % I think this gives me some symbols
 \usepackage{amsthm}        % Does theorem stuff
% \usepackage{amssymb}       % more symbols and fonts
%  \usepackage{accents}
%  \usepackage{bbm}           % for the nice blackboard bold 1
 \usepackage{empheq}        % Some more extensible arrows, like \xmapsto
 \usepackage{enumitem}
 \usepackage{mathrsfs}      % Sheafy font \mathscr{}
% \usepackage{picinpar}      % for pictures in paragraphs
% \usepackage{tikz}
 \usepackage[nofancy]{rcsinfo}
 \usepackage[all]{xy}       % Include XY-pic
    \SelectTips{cm}{10}     % Use the nicer arrowheads
    \everyxy={<2.5em,0em>:} % Sets the scale I like
 \usepackage[pdfauthor={Wafa BENKAOUAR},%
pdftitle={Multimodal detection of Intention of Interaction},%pagebackref=true
]{hyperref}


%
\hypersetup{
  colorlinks=true,
  linkcolor=black,
  %linktoc=none,
  citecolor=red,
  %Mpagebackref=true,
urlcolor=cyan,
  bookmarksnumbered=true}

%

%%%%%%% Stuff for keeping track of sections %%%%%%%%%%%%%%%%%%%%%%%%%%%%
 \newcount\n
 \newcommand{\sektion}[2]{\stepcounter{section} \renewcommand{\thesection}{#1}\n=\count0 \newpage
\ifnum\count0=\n \ifnum\count0>1\ \newpage \fi\fi\section{#2} \gdef\sectionname{#1\quad #2, v.~\rcsInfoMonth-\rcsInfoDay}}
 \newcommand{\subsektion}[1]{\subsection*{#1} \addcontentsline{toc}{subsection}{#1}}
 %% This is the empty section title, before any section title is set %%
 \newcommand\sectionname{}                                           %%
%%%%%%% End stuff for keeping track of sections %%%%%%%%%%%%%%%%%%%%%%%

%%%%%%%%%%%%%%% Theorem Styles and Counters %%%%%%%%%%%%%%%%%%%%%%%%%%
 \renewcommand{\theequation}{\thesection.\arabic{equation}}         %%
 \makeatletter                                                      %%
    \@addtoreset{equation}{section} % Make the equation counter reset each section
    \@addtoreset{footnote}{section} % Make the footnote counter reset each section
                                                                    %%
 \newenvironment{warning}[1][]{%                                    %%
    \begin{trivlist} \item[] \noindent%                             %%
    \begingroup\hangindent=2pc\hangafter=-2                         %%
    \clubpenalty=10000%                                             %%
    \hbox to0pt{\hskip-\hangindent\manfntsymbol{127}\hfill}\ignorespaces%
    \refstepcounter{equation}\textbf{Warning~\theequation}%         %%
    \@ifnotempty{#1}{\the\thm@notefont \ (#1)}\textbf{.}            %%
    \let\p@@r=\par \def\p@r{\p@@r \hangindent=0pc} \let\par=\p@r}%  %%
    {\hspace*{\fill}$\lrcorner$\endgraf\endgroup\end{trivlist}}     %%
                                                                    %%
 \newenvironment{exercise}[1][]{\begin{trivlist}%                   %%
    \item{\bf Exercise\@ifnotempty{#1}{ #1}. }\it}{\end{trivlist}}   %%
 \newenvironment{solution}{\begin{trivlist}%                        %%
    \item{\it Solution.}}{\end{trivlist}}                           %%
                                                                    %%
 \def\newprooflikeenvironment#1#2#3#4{%                             %%
      \newenvironment{#1}[1][]{%                                    %%
          \refstepcounter{equation}                                 %%
          \begin{proof}[{\rm\csname#4\endcsname{#2~\theequation}%   %%
          \@ifnotempty{##1}{\the\thm@notefont \ (##1)}\csname#4\endcsname{.}}]%%
          \def\qedsymbol{#3}}%                                      %%
         {\end{proof}}}                                             %%
 \makeatother                                                       %%
                                                                    %%
 \newprooflikeenvironment{definition}{Definition}{$\diamond$}{textbf}%
 \newprooflikeenvironment{example}{Example}{$\diamond$}{textbf}     %%
 \newprooflikeenvironment{remark}{Remark}{$\diamond$}{textbf}       %%
                                                                    %%
 \theoremstyle{plain}                                               %%
 \newtheorem{theorem}[equation]{Theorem}                            %%
 \newtheorem*{claim}{Claim}                                         %%
 \newtheorem*{lemma*}{Lemma}                                        %%
 \newtheorem*{theorem*}{Theorem}                                    %%
 \newtheorem{lemma}[equation]{Lemma}                                %%
 \newtheorem{corollary}[equation]{Corollary}                        %%
 \newtheorem{proposition}[equation]{Proposition}                    %%
%%%%%%%%%%% End Theorem Styles and Counters %%%%%%%%%%%%%%%%%%%%%%%%%%

%%% These three lines load and resize a caligraphic font %%%%%%%%%
%%% which I use whenever I want lowercase \mathcal %%%%%%%%%%%%%%%
 \DeclareFontFamily{OT1}{pzc}{}                                 %%
 \DeclareFontShape{OT1}{pzc}{m}{it}{<-> s * [1.100] pzcmi7t}{}  %%
 \DeclareMathAlphabet{\mathpzc}{OT1}{pzc}{m}{it}                %%
                                                                %%
%%% and this is manfnt; used to produce the warning symbol %%%%%%%
 \DeclareFontFamily{U}{manual}{}                                %%
 \DeclareFontShape{U}{manual}{m}{n}{ <->  manfnt }{}            %%
 \newcommand{\manfntsymbol}[1]{%                                %%
    {\fontencoding{U}\fontfamily{manual}\selectfont\symbol{#1}}}%%
%%%%%%%%%%%%%%%%%%%%%%%%%%%%%%%%%%%%%%%%%%%%%%%%%%%%%%%%%%%%%%%%%%
%%%%%%%%%%%%%%%%%%%%%%%%%%%%%%%%FONT%%%%%%%%%%%%%%%%%%%%%%%%%%%%%
%\usepackage[urw-garamond]{mathdesign}
\usepackage[charter]{mathdesign}
%\usepackage{mathptmx}			
%\usepackage{pxfonts}
\usepackage[T1]{fontenc}		
%\linespread{1.05}         % Palatino needs more leading (space between lines)
%%%%%%%%%%%%%%%%%%%%%%%%%%%%%%%%%%%%%%%%%%%%%%%%%%%%%%%%%%%%%%%%%
\usepackage{graphicx}
\usepackage{caption}
\usepackage{subcaption}
\usepackage{setspace}
\usepackage{multirow}
\usepackage[table]{xcolor}
\setlength{\parindent}{0.4cm}
% \usepackage{pgfplots}
\usepackage{bchart}
