% !TEX TS-program = pdflatex
% !TEX encoding = UTF-8 Unicode

% This is a simple template for a LaTeX document using the "article" class.
% See "book", "report", "letter" for other types of document.

\documentclass[11pt]{article} % use larger type; default would be 10pt

\usepackage[utf8]{inputenc} % set input encoding (not needed with XeLaTeX)
\usepackage{hyperref}
%%% Examples of Article customizations
% These packages are optional, depending whether you want the features they provide.
% See the LaTeX Companion or other references for full information.

%%% PAGE DIMENSIONS
\usepackage{geometry} % to change the page dimensions
\geometry{a4paper} % or letterpaper (US) or a5paper or....
% \geometry{margin=2in} % for example, change the margins to 2 inches all round
% \geometry{landscape} % set up the page for landscape
%   read geometry.pdf for detailed page layout information

\usepackage{graphicx} % support the \includegraphics command and options

% \usepackage[parfill]{parskip} % Activate to begin paragraphs with an empty line rather than an indent

%%% PACKAGES
\usepackage{booktabs} % for much better looking tables
\usepackage{array} % for better arrays (eg matrices) in maths
\usepackage{paralist} % very flexible & customisable lists (eg. enumerate/itemize, etc.)
\usepackage{verbatim} % adds environment for commenting out blocks of text & for better verbatim
\usepackage{subfig} % make it possible to include more than one captioned figure/table in a single float
% These packages are all incorporated in the memoir class to one degree or another...

%%% HEADERS & FOOTERS
\usepackage{fancyhdr} % This should be set AFTER setting up the page geometry
\pagestyle{fancy} % options: empty , plain , fancy
\renewcommand{\headrulewidth}{0pt} % customise the layout...
\lhead{}\chead{}\rhead{}
\lfoot{}\cfoot{\thepage}\rfoot{}

%%% SECTION TITLE APPEARANCE
\usepackage{sectsty}
\allsectionsfont{\sffamily\mdseries\upshape} % (See the fntguide.pdf for font help)
% (This matches ConTeXt defaults)

%%% ToC (table of contents) APPEARANCE
\usepackage[nottoc,notlof,notlot]{tocbibind} % Put the bibliography in the ToC
\usepackage[titles,subfigure]{tocloft} % Alter the style of the Table of Contents
\renewcommand{\cftsecfont}{\rmfamily\mdseries\upshape}
\renewcommand{\cftsecpagefont}{\rmfamily\mdseries\upshape} % No bold!

%%% END Article customizations

%%% The "real" document content comes below...

\title{Compte-rendu de R\'eunion du 30 novembre 2012}

%\date{} % Activate to display a given date or no date (if empty),
         % otherwise the current date is printed 

\begin{document}
\maketitle

Ordre du Jour 
\begin{itemize}
\item Presentation du siteweb, remarques et ameliorations
\item Propositions de senarios
\item Discussion sur les règles de publications
\end{itemize}



\paragraph{Presentation du siteweb, remarques et ameliorations}
demande acces admin pout tous \\
creation des rubriques deja faites \\
envoie de mail quand ajout \\
demande d'ajout d'une rubrique prive pour partage de doc entre nous \\
evocation de mendeley : ouvrir le groupe lig à tous mocas
\paragraph{Propositions de senarios}
LIG : 5 mini scenettes  derives des 5 roles \\
LIMSI : interaction plusieurs enfants et plusieurs robots (acces a classe et centre aere pour experiementation) \\
jeu taboo, jeu de role (tournoi pour faire duree dans la longueur et avoir un effet relationel) jeu cooperation et competion entre robot et enfant. Imagine une scenes ou un enfant apporterai son robot chez un ami pour jouer.  \\
Presentation de theorie deriver du busines  avec une boucles : sollicitation, accompagnmenet, evaluation de l'engagement, emprise, autre tache  Brangier \url{http://www.univ-metz.fr/ufr/sha/2lp-etic/Criteres_Persuasion_Interactive-2.pdf}


\paragraph{Discussion sur les règles de publications et autres}
Proposition de prendre un stagiaire en ecole de cine pour faire la video finale documantaire du projet (Magalie Ochs, propose de voir sur paris)
Publication d'un papier commun au 4 equipes tous les ans avec management du papier en rotation sur les equipes. 

\paragraph{prochaine reunion au limsi le 22 ou 25 janvier}


\end{document}
