\chapter{BRAHMS Modelling}
In this chapter we describe the specification foe the modelling of the previously described scenarios on the BRAHMS plateform \cite{BrahmUserGuide}. 

\section{Scenario in a time line}
The scenario is here describe roughly. We describe the events that occurs and the view point is the companions' one. 
%The workflow is presented in the picture \ref{fig:figure1}. 
\paragraph{}
It's 6pm, Ben is in the living room watching tv, and Prof signals him that it is time for homework.
Prof motivates Ben to do his homework then ask him to go to the desk and get his textbook, while prof checks the agenda on the webpage of the school. 
Prof schedules the different exercises and communicate the schedule to Ben.
Prof load the first activity from the Homework\-AppStore.
\paragraph{}
Then if time has come to stop the homework (30 minutes after and that Ben has completed everything), Prof announce that the time for homework is over, if the homework were not completed Prof proposes to go back on them later.
Seeing that Ben is free, Buddy asks Ben if he is bored. If yes Buddy incites him to play some imitation game.
\paragraph{}
While Buddy and Ben are playing, the bell rings. Bodyguard asks Buddy to suspend the game and checks who is at the door. Bodyguard reassure Ben and informs him that it is his friend Alan who took a companion Poto with him. 
Bodyguard open the doors for Alan. Since both Alan and Ben have their companions Buddy proposes to play a taboo game. 
\paragraph{}
When the game is over, Ben feels a bit upset because he lost.
Dolly comforts him and plays Ben favourite song to cheer him up.
\paragraph{}
At 19h15pm Bodyguard informs Alan that he should go home (send a message to his parents to tell them he is on his way home). 
\paragraph{}
If all the homework are done, Coach proposes Ben to learn a play (because Ben likes theatre). 
\paragraph{}
Finally at 19h45 the parents are back and Buddy and Ben can tell them what they did. 
\section{Agents}
\section{Objects}
\section{Geography}
