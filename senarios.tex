\chapter{Scenarios}

This document is a first draft for the formalization of the scenarios in the frame of the MOCA ANR Project. 
The level of abstraction to describe the scenarios has been taken high. Other documents will follow with lower level of abstraction. 

\section{General Background}
Needs of children and teenagers are defined by the OMS and some risks of mental disorder appearance are pointed in a report %\cite{oms}. 
They specially point at the fact that some social factors could originate mental disorders at that age. Mainly reward and valorization from family, school and social implications are presented as protective measures against these risks. They also advocate cultural experiences, and models of successful behaviours as preventive measures.
The needs and rights of children and teenagers are known as security, leisure and rest, need to measure the risks, need of socialization, need of autonomy, need of creativity and imagination, need of imitation, need to act and to feel.

\section{Scenario 1 : Small scenes}
This part aims to present the different little scenarios for the MOCA project screening the different roles of the companions of the little world. 

\paragraph{Background}
This first part titled background set up the need and the problem that incite the role of the companion for the child.
It puts the context of this project. 
\paragraph{Abilities}
In all of these context a need of incitations (motivation) and valorization will be emphasize. 


\subsection{Act I : Teacher }
\paragraph{Background}
More and more both of the parents work and a child or a young teenage found himself quite often \textbf{alone home} after school to manage himself. 
This period can also be really \textbf{stressful} for the kids who might need also someone to reassure them and help them with their confidence. 
Regarding the homework, the child who used to be supported by its parents during this task, has now to face it alone. 
Beside, the parents might like to still \textbf{keep an eye on the progress} of their kids. 
\paragraph{Abilities}

\begin{itemize}
	\item Should have the knowledge and the expertise required to help the child in the homework task.
	\item Should motivate and reward good performance, while criticise and discourage on a bad one. 
	\item Should be able to interpret the mental state of the child to give him appropriate feedback or treatment.
	\item Should be able to track the performance of the child over time to monitor his development progress and adjust pedagogic parameters.
	\item Should be able to summarize daily work accomplished with the child.
\end{itemize}



\subsection{Act II : Bodyguard}
\paragraph{Background}
A child alone home after school might face some stressful situations were advices could be useful as for example when a stranger rings the bell. 
In all of these context a need of incitations (motivation) and valorization will be emphasize. 
\paragraph{Abilities}
\begin{itemize}
\item Should be able to start an alarm
\item Should be able to call the parents, firemen, police
\item Should reassure the child
\item Should give advices on how to react to a situation
\end{itemize}


\subsection{Act III : Playmate}
\paragraph{Background}
Boredom can also be experience by a lonely kid home. 
It would be better to stimulate its creativity. 
\paragraph{Abilities}
\begin{itemize}
\item Should propose stimulating games
\item Should propose and play : gave for creativity and imagination
\item Could joke (Carambar jokes)
\end{itemize}

\subsection{Act IV : Comforter}
\paragraph{Background}
The blues
\paragraph{Abilities}
\begin{itemize}
\item Should be able to detect the mood of the child (alert parents in case)
\item Should listen and give advices
\end{itemize}


\subsection{Act V : Coach}
\paragraph{Background}

Need to open to new things, extra school activities, good for development of the child. 
Cognitive needs of the children. 
Obesity and need for physical activities
\paragraph{Abilities}
The coach abilities can be adapted to the wishes of the parents and the child, in order to get a different knowledge than the one taught at school.
The coach can pass this knowledge by participative activities of supervises activities. \\
In the frame of the participative activities the coach would be able to : 
\begin{itemize}
\item Should be able to speak and teach foreign languages
\item Should incite physical activities (dancing, zumba)
\end{itemize}
In some other cases the coach would be a supervisor of the activities, and be able to : 
\begin{itemize}
\item Should give instruction to cook a meal in security
\item Should be able to supervise on musical practice
\end{itemize}

\subsection{Storyboard}
\subsubsection{The different sequences of activities for each role}
\paragraph{Teacher} \\
\textbf{Start} : Time, it's 18h, time for homeworks \\
\textbf{Action Sequence} : 
\begin{itemize}
\item[Act1] : Signal the time to Ben "time for homeworks" \- motivation, incitation
\item[Act2] : Ask to go to the desk
\item[Act3] : Check Agenda
\item[Act4] : Start from HMK\_Activity\_Store
\item[Act5] : Finish the activity, valorization, report parent
\end{itemize}



\subsubsection{The environment}
There are 6 locations involved in our world : Ben's house, the school, the Streets, the Kid centre, Papa's work place, the Mom's work place. 
\paragraph{Ben's house}
The house is located on Maple Street, in the old town corner.
The house is on 2 floors. 
On the ground floor there is the kitchen, the living room and the office of Ben's father.
On the upper floor , Ben's bedroom, his parents bedroom and the bathroom. 


\subsubsection{The agents}
Ben is 11 years old. He is in his first year of middle school. He has a father and a mother who both work until 8PM. Ben finishes school at 4.30PM. The school is just few streets away from home and Ben usually walks with some friends to come home every evening. 
Ben has usually homework to do every evening. His grades at school are average and with more efforts his parents and his teacher Miss Jones. Ben knows that he should do his homework but he prefers playing to video games on-line. On Saturday he usually goes to the kid centre to play with Alan and its companions at Taboo. 
Ben is a fan of Manga and recently started to learn Japanese with Coach.
When Ben gets hungry he usually goes to the kitchen and get some snack He prefers to have a chocolate bar rather than an apple.
The Coach proposes also some hip hop tips as Ben likes Hip Hop and needs to exercise. 
The house rule is that before playing any game or watching tv, Ben should complete his homework. 
Prof proposes is help for the homework and inform the cloud when the homework are complete. 
Prof also gets information from the parents and the Miss Jones. These information are related to the subject that have to be particularly studied by Ben. Prof makes a synthesis of the work accomplish by Ben daily and can give a feedback to the parents of Ben or to the teacher if they ask for it. 
The results are added to the diary of Ben that is managed by the Cloud. 
The Prof encourages Ben to do his homework with care.
When the Prof encourages Ben, his motivation increases and he belief more that he can complete the task. 

% Soon as she starts to read the first few problems, Johnny, the house robot enters her room watching her attempting homework. He greets her and offers to help her. He asses her assignment and tries to offer her motivation and encouragement. Soon after, Sarah feels more confident and relaxed to continue working, demanding Johnny time to time for help with the assignment. Johnny finishes with a pat on Sarah's back, telling her that she did a good job with a smile, while making a careful assessment of Sarah's performance and generating and emailing a report for the parents.


\section{Scenario 2 : Multi-agents highlight}
This scenario has been derived from the proposition by Dominique Duhaut 
\paragraph{Background}
The child wants to watch tv, but now it is time for homework.
\paragraph{Abilities}

\subsection{Storyboard}

\section{Scenario 3 : Multi-agent and Multi-user}
\paragraph{Background}
Playing a game with robots and humans as companions. 
Each child with his companion playing Taboo. 
 \paragraph{Abilities}


\subsection{Storyboard}



